% Options for packages loaded elsewhere
\PassOptionsToPackage{unicode}{hyperref} \PassOptionsToPackage{hyphens}{url}
%
\documentclass[ ]{article}
\usepackage{amsmath,amssymb}
\usepackage{iftex}
\ifPDFTeX
  \usepackage[T1]{fontenc}
  \usepackage[utf8]{inputenc}
  \usepackage{textcomp} % provide euro and other symbols
\else % if luatex or xetex
  \usepackage{unicode-math} % this also loads fontspec
  \defaultfontfeatures{Scale=MatchLowercase}
  \defaultfontfeatures[\rmfamily]{Ligatures=TeX,Scale=1}
\fi
\usepackage{lmodern}
\ifPDFTeX\else
  % xetex/luatex font selection
\fi

\usepackage[sorting=none]{biblatex}
\addbibresource{bibliography.bib}
% Use upquote if available, for straight quotes in verbatim environments
\IfFileExists{upquote.sty}{\usepackage{upquote}}{}
\IfFileExists{microtype.sty}{% use microtype if available
  \usepackage[]{microtype}
  \UseMicrotypeSet[protrusion]{basicmath} % disable protrusion for tt fonts
}{}
\makeatletter
\@ifundefined{KOMAClassName}{% if non-KOMA class
  \IfFileExists{parskip.sty}{%
    \usepackage{parskip}
  }{% else
    \setlength{\parindent}{0pt}
    \setlength{\parskip}{6pt plus 2pt minus 1pt}} }{% if KOMA class
  \KOMAoptions{parskip=half}}
\makeatother
\usepackage{xcolor}
\setlength{\emergencystretch}{3em} % prevent overfull lines
\providecommand{\tightlist}{%
  \setlength{\itemsep}{0pt}\setlength{\parskip}{0pt}}
\setcounter{secnumdepth}{-\maxdimen} % remove section numbering
\ifLuaTeX \usepackage{selnolig}  % disable illegal ligatures
\fi
\IfFileExists{bookmark.sty}{\usepackage{bookmark}}{\usepackage{hyperref}}
\IfFileExists{xurl.sty}{\usepackage{xurl}}{} % add URL line breaks if available
\urlstyle{same}
\hypersetup{ hidelinks, pdfcreator={LaTeX via pandoc}}

\author{}
\date{}

\begin{document}

\hypertarget{uoe-unaherdoc-proposal}{%
\section{UoE UnaHerDoc Proposal}\label{uoe-unaherdoc-proposal}}

\hypertarget{overview}{%
\subsection{Overview}\label{overview}}

\begin{itemize}
\tightlist
\item
  rough history
\end{itemize}

Laser vibrometery has long been a useful tool for for acoustic analysis
\{SRIRIAM}. With respect to musical insrtuments, Modes give an understanding of
material frequency response, mode shapes \cite{SKRODZKA, MOYNE}, but also
potentially provide a means of estimating material properties
\cite{DUCCESCHI2024109949}.

One of the methods of measure is via laser vibrometry, which has been a useful
tool for acousticians for a number of decades. Laser vibrometry provides both
accurate measurement but is also non-obtrusive. Laser vibrometry does have a
limitation in the form of cost. A modest laser doppler vibrometry would be in
the order of €20k. This is a cost that is difficult to justify for the usage it
would see by instrument collector and museums. The market for laser vibrometer
is also domniated by a handful of companies and it seems unlikely this cost will
shrink any time soon. Laser vibrometers are precision engineered for all
vibration analysis, a level accuracy unlikely to be utlilsed when applied to
musical instrumebts. The thesis by Malah \cite{MALAHS2015DESIGNOA} outlines such
a system for a Grazing Laser Vibrometer. This project would expand Malah's work
by applying it to a Laser Doppler Vibrometer as well as automating the process
with respect to soundboard measurments. 

Estimation of material properties of plates  without the additional cost of
Finite Element Method modelling software.


Project would aim to explore automation of the process the benefits of which are
two-fold. Firstly, automation would allow the measurement process to take place
without manual operation saving time in labour and potentially time in resetting
each measurement. Secondly, an automated proces would make measurements more
easily reproduecable. Multiple rounds of measurmenets could be taken and the
accuracy of the experimental setup could be more easily tested. Linear scanning
using a standard 2 dimensional plotter yielded high-resolution results in
\cite{HUI}. Such a measurement arrangement would allow for automated scanning.
The 3D printed impact hammer used for measurements  in \{RAU} can also be
combined with an actuator (e.g. a solenoid) for reproducible excitation.

![image of 2D plotted from \cite{HUI}](.img/<>)

The barrier to entry for technical projects of this kind has been lowered over
the past few decades. What was once quite a technical undertaking is now more
achievable with readily available micro-controller units (MCUs) like the Arduino
or single-board computers (SBCs) like the Raspberry Pi, which can easily
interface with sensors and actuators to prototype haptic devices. This proposal
will cover what components will be required in the project and what function
they will serve.

Goal of the project would be the design of an automated non-contact measurement
system for musical instrument soundboards. The system would aim to be economic
to fabricate in comparison to current commercially available laser vibrometers.


The output would consist of:

\begin{itemize}
\tightlist
\item
  An open source, version controlled framework for designing a low-cost,
  non-contact measurement system for musical instrument soundboards consisting
  of:

  \begin{itemize}
    \tightlist
    \item
      Operation Repository: An open source repository containg:
      
      \begin{itemize}
        \tightlist
        \item
        Instructions for self-assembly of measurement tools and experiment setup 
        \item
        CAD models for digital fabrication of components and PCB
        \item
        Control software for automation of a measurement rig
        \item
        Analysis software for recording, processing and analysing measurement
        data.
      \end{itemize}
    
    \item
    Technical Documentation: A document containing: , justification of design
    decisions, discussion of formulae required to obtain detain from the system, 
    \begin{itemize}
      \tightlist
      \item
      An outline of the design process
      \item
      Technical background and justifications for any design decision and
      components utislised    
      \item
      Design modifications, their relative cost to the project and impact on
      results. 
    \end{itemize}
  \end{itemize}




Laser Doppler virbo

\hypertarget{recent-research}{%
\subsection{Recent Research}\label{recent-research}}

\begin{itemize}
\tightlist
\item
  what has been in the field
\item
  what is the trajectory
\item
  where is there room for improvement
\end{itemize}

Matiss Malahs researched low-cost laser vibrometer system for a PhD thesis
\cite{MALAHS2015DESIGNOA}. The thesis is an overview of interferometry, laser
doppler vibrometers and Grazing laser vibrometers as well as an instruction on
fabricating a GLV economically. This project would expand on Matiss' work by
exploring the construction of both affordable LDV and GLV measurmenet units, in
particular with an application to soundboards. Measuring musical instrument
soundbard does not require as wide a frequency bandwidth compared to other
vibration analysis applications. The polytec PSV400  has bandwidth up to 1000kHz
\cite{PSV400}, compare to soundboards where only up to 20kHz are consideration
for measurement as it is the limit of human hearing.

The paper Hui et al. \cite{HUI} considers the addition of a linear scanning unit
paired with a laser vibrometer. In this paper high-resolution results were
obtained. This project would expand by applying the same methodology to an
economic laser vibrometer assembly.

For instrument restoration such as demonstrated in  \cite{MOYNE}, laser
vibrometery is a tool for being able to recreate an instrument. For such a
use-case, the project would aim to provide an estimation of the material
properties which can be used cross referenced with material properties found in
literature.

An array of MEMs microphone has been considered as an option for non-contact
economic modal analysis. This is a promiosing apporach for ovtaining FRFs as
outlined in \cite{Farshidi} and may be accurate enough for estimating material
properties. MEMs microphone arrays as a means of plate mode measurement is
deomnstrated in \cite{VELSEN} stating that for a MEMs microphone array:

\begin{quote}
low frequency domain eigenfrequencies of a plate-like structure can be
identified with high accuracy. Identification of the mode shapes and the damping
constants are shown not to be accurate.
\end{quote}

Since the focus of the project is on derivation of mode frequencies and shapes
exploration specificallly is not a goal of the project, the usage of MEMs
microphone arrays has not been considered.

\cite{RAU}

\hypertarget{project-description}{%
\subsection{Project Description}\label{project-description}}

Look at three methods for measurement.

\begin{itemize}
\tightlist
\item
  Laser Doppler
\item
  Grazing Laser
\item
  MEMs microphone array
\end{itemize}

For each there is a base cost.

Look at a simple version for each and how promising it is. proceed with
whichever methods showed the most promise.

This section lay out objectives, give a rough timeline and stipulate which
materials are required along with a rough costing.

\hypertarget{objectives}{%
\subsubsection{Objectives}\label{objectives}}

\begin{itemize}
\tightlist
\item
  what hope to achieve

  \begin{itemize}
  \tightlist
  \item
    a framework for designing a low-cost, non-contact measurement system for
    musical instrument soundboards.

    \begin{itemize}
    \tightlist
    \item
      comprehensive instructions for self assembly
    \item
      cad models for fabrications
    \item
      document outlining:

      \begin{itemize}
      \tightlist
      \item
        the design process
      \item
        justification for any design decision
      \item
        technichal background to make modification in the design easier to
        undertake.
      \end{itemize}
    \end{itemize}
  \item
    open source software toolkit for

    \begin{itemize}
    \tightlist
    \item
      control
    \item
      data processing
    \end{itemize}
  \end{itemize}
\end{itemize}

\hypertarget{timeline}{%
\subsubsection{Timeline}\label{timeline}}

\begin{itemize}
\tightlist
\item
  how long will each section take.
\end{itemize}

\hypertarget{materials}{%
\subsubsection{Materials}\label{materials}}

\begin{itemize}
\tightlist
\item
  What will be needed
\end{itemize}

\hypertarget{general-electronics}{%
\paragraph{General electronics}\label{general-electronics}}

Carrying out a project like this will require a workshop space with facilities
applicable to all approaches to measurements techniques. In addition to access
to general electronic sundries (resistors, capacitors, potentiometers), workshop
would require:

\begin{itemize}
\tightlist
\item
  Benchtop Power Supply: Parametric power supply required for prototyping
  electronics when power requirements are still in flux..
\item
  Function generator: For applying a functional signal to a circuit for
  simulating input from sensors.
\item
  Oscilloscope: Signal measurement and testing
\item
  Soldering Station: For soldering though-hole and surface mount components.
  (optionally) A reflow oven for soldering surface mount components to a pcb.
  Alternatively this step could be carried out from an external service.
\end{itemize}


\hypertarget{laser-components}{%
\paragraph{Laser Components}\label{laser-components}}


  \begin{itemize}
  \tightlist
  \item
    LM348 op-amps: used in the Transimpedance Amplifier as well as filtering
    circuits, and level shifting. As recommended in \cite{MALAHS2015DESIGNOA} an
    alternative would aso be sought to improve sensor bandwidth.
  \item
    650nm Laser: Bandwidth chosen in \cite{MALAHS2015DESIGNOA} to match the
    BPW34 photodiodes. The BPW3 is shown to have the best response in the
    Infrared range so a dual laser system, one for aiming one for measuring,
    would be desirable.
  \item
    BPW34 photodiodes: Used for detecting changes in the laser beam. Current is
    induced when the light strikes the surface.
  \item
    Analog to Digital Converter: The GLV from \cite{MALAHS2015DESIGNOA} was
    connected to a National Instruments myDAQ which provided enough resolution
    for processing data. As such, an ADC of similar specifications would be
    desirable.
  \end{itemize}

\hypertarget{ldv}{%
\subparagraph{LDV}\label{ldv}}

\begin{itemize}
\tightlist
\item
  beam splitter(s)
\item
  mirror
\item
  Bragg Cell

  \begin{itemize}
  \tightlist
  \item
    fibre optics
  \item
    From Matiss Malahs \emph{If the information about the direction in which the
    object is moving is not necessary, the Bragg cell can be omitted. {[}9.{]}}
  \end{itemize}
\item
  Questions

  \begin{itemize}
  \item
    x3 beam splitters
  \item
    mirror
  \item
    brag cell in the order of £1500

    \begin{itemize}
    \tightlist
    \item
      the cell
    \item
      fibre optics
    \item
      focal lens
    \end{itemize}
  \item ~ \hypertarget{digital-signal-processing}{%
    \subsection{digital signal processing}\label{digital-signal-processing}}
  \end{itemize}
\item
  Linear Scanning rig

  \begin{itemize}
  \tightlist
  \item
    steppers
  \item
    arduino
  \item
    extrusion
  \end{itemize}
\end{itemize}

\hypertarget{safety}{%
\subsubsection{Safety}\label{safety}}

Since the project will be working with lasers there are safety concerns. The
design in \cite{MALAHS2015DESIGNOA} was Category 3R laser with a maximum power
of 5mW at 3V.

\begin{quote}
Class 3R lasers are specified as to create no risk to skin and low risk to eyes
with power limit up to 5 mW [\cite{SAFETY}]
\end{quote}

Eye protection would need to be worn and appropriate signage displayed in any
workshop space.

The designs considered would initially use a commercial power supply unit or
battery power supply units. As such, power electronic design is not considered
as part of the design process and no high-voltage power electronic design will
be taking place.


\hypertarget{outcomes}{%
\subsubsection{Outcomes}\label{outcomes}}

\subsubsection{Stage One: Measurment unit prototypoing}

The goal of this stage is to create a LDV and GLV unit. The LDV would not
contain a Bragg Cell and the viability of the system can be tested against the
GLV.

\subsubsection{Stage Two: Linear Scanning}

Both LDV and GLV units will be tested when mounted on a linear scanning system.

Main goal is to create a means of stipulating geomerty and asses each unit
against a simple rectangular plate.

\subsubsection{Stage Three: Documentation}

The final stage will hope to generate a document which will facilitate future
modal analysis Projects. Focus of this stage is the collation of a written
thesis, technical documentation and an open source hardware and software
repository, with a focus on software sustainability to make the project more
easily developed and reproducible in the future.

\begin{itemize}

\tightlist
\item
  Phases of the project

  \begin{itemize}
  \tightlist
  \item
    First sensor fabrications

    \begin{itemize}
    \tightlist
    \item
      testing / reproducibility
    \item
      cost tracking
    \end{itemize}
  \item
    automation
  \end{itemize}
\end{itemize}

\end{document}
