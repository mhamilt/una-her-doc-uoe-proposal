% Options for packages loaded elsewhere
\PassOptionsToPackage{unicode}{hyperref} \PassOptionsToPackage{hyphens}{url}
%
\documentclass[ ]{article}
\usepackage{amsmath,amssymb}
\usepackage{iftex}
\ifPDFTeX
  \usepackage[T1]{fontenc}
  \usepackage[utf8]{inputenc}
  \usepackage{textcomp} % provide euro and other symbols
\else % if luatex or xetex
  \usepackage{unicode-math} % this also loads fontspec
  \defaultfontfeatures{Scale=MatchLowercase}
  \defaultfontfeatures[\rmfamily]{Ligatures=TeX,Scale=1}
\fi
\usepackage{lmodern}
\ifPDFTeX\else
  % xetex/luatex font selection
\fi

% Bibliography Package and Setup
\usepackage[sorting=none]{biblatex}
\addbibresource{bibliography.bib}

\usepackage{graphicx}
\usepackage{subfig}

% Use upquote if available, for straight quotes in verbatim environments
\IfFileExists{upquote.sty}{\usepackage{upquote}}{}
\IfFileExists{microtype.sty}{% use microtype if available
  \usepackage[]{microtype}
  \UseMicrotypeSet[protrusion]{basicmath} % disable protrusion for tt fonts
}{}
\makeatletter
\@ifundefined{KOMAClassName}{% if non-KOMA class
  \IfFileExists{parskip.sty}{%
    \usepackage{parskip}
  }{% else
    \setlength{\parindent}{0pt}
    \setlength{\parskip}{6pt plus 2pt minus 1pt}} }{% if KOMA class
  \KOMAoptions{parskip=half}}
\makeatother
\usepackage{xcolor}
\setlength{\emergencystretch}{3em} % prevent overfull lines
\providecommand{\tightlist}{%
  \setlength{\itemsep}{0pt}\setlength{\parskip}{0pt}}
\setcounter{secnumdepth}{-\maxdimen} % remove section numbering
\ifLuaTeX \usepackage{selnolig}  % disable illegal ligatures
\fi
\IfFileExists{bookmark.sty}{\usepackage{bookmark}}{\usepackage{hyperref}}
\IfFileExists{xurl.sty}{\usepackage{xurl}}{} % add URL line breaks if available
\urlstyle{same}
\hypersetup{ hidelinks, pdfcreator={LaTeX via pandoc}}

\title{Automated Non-Contact Measurement System Design For Musical Instrument Soundboards}
\author{Matthew Hamilton}
\date{\today}

\begin{document}

\begin{titlepage}
\maketitle

\begin{figure}
\centering
\includegraphics[width=0.66\textwidth]{./img/mode-shape.png}
\label{fig:mode-shape}
\end{figure}
\end{titlepage}

\hypertarget{overview}{%
\section{Overview}\label{overview}}

Modal analysis provides an understanding of material frequency response,
vibration patterns through mode shapes \cite{SKRODZKA, MOYNE}, and can also
provide a means of estimating material properties \cite{DUCCESCHI2024109949}.
Risk to an instrument is prominent concern during analysis, in particular to
historic musical instruments. This has lead to the emergence of non-contact
methods for measuring vibration. These new methods are appealing for modal
analysis for fragile or expensive instruments. 

Laser vibrometry has long been a useful tool for acoustic analysis
\cite{SRIRIAM}, especially with respect to musical instruments. Laser vibrometry
provides a non-obtrusive means of obtaining accurate results, however it does
have a limitation in the form of cost. A modest Laser Doppler Vibrometer (LDV)
cost in the order of €20k. In addition, estimation of material properties of
soundboards comes with an additional cost of Finite Element Method modelling
software. This is a cost that is difficult to justify for the usage it would see
by instrument collector and museums. The market for laser vibrometer is also
dominated by a handful of companies and it seems unlikely this cost will shrink
any time soon. Laser vibrometers are precision engineered for all vibration
analysis, a level accuracy unlikely to be utilised when applied to musical
instruments. 

The thesis by Malah \cite{MALAHS2015DESIGNOA} outlines such a system for a
Grazing Laser Vibrometer. The proposed project would expand Malah's work by
applying it to a Laser Doppler Vibrometer. In addition, the project will explore
automating the process of measurement and provide an open source framework for
analysing result.

\begin{figure}
  \centering
  \includegraphics[width=0.66\textwidth]{./img/ldv-setup.png}
  \caption{LDV Measurement Setup in \cite{MANSOUR}}
  \label{fig:ldv-setup}
\end{figure}

  
The benefits of automating the measurement process for modal analysis are
two-fold. Firstly, automation would allow the entire process to take place with
minimum manual intervention, saving time in labour and potentially time in
carrying out measurements. Secondly, an automated process would make
measurements more easily reproducible. Multiple rounds of measurements could be
taken and the accuracy of the experimental setup could be more easily tested.
The project aims to explore viable, affordable and reproducible means for
augmenting a laser vibrometer to a scanning laser vibrometer.

The barrier to entry for technical projects of this kind has been lowered over
the past few decades. What was once quite a technical undertaking is now more
achievable with readily available micro-controller units (MCU) like the Arduino
or single-board computers (SBC) like the Raspberry Pi, which can easily
interface with sensors and actuators to prototype haptic devices. This proposal
will cover what components will be required in the project and what function
they will serve.

\begin{figure}
\centering
\includegraphics[width=0.66\textwidth]{./img/guitar-mode-table.png}
\caption{Table of guitar body mode shapes derived from LDV in \cite{SKRODZKA}}
\label{fig:guitar-mode-table}
\end{figure}

\hypertarget{recent-research}{%
\subsection{Recent Research}\label{recent-research}}

This section will discuss some projects from the last 15 years focused on the
topics of non-contact, automated modal analysis and making the process more
economically accessible. I will then elaborate on how the proposed projects
intends to enhance and build on these projects. 

Matiss Malahs researched low-cost laser vibrometer system for a PhD thesis
\cite{MALAHS2015DESIGNOA}. The thesis is an overview of interferometry, LDVs and
Grazing Laser Vibrometers (GLV) as well as an instruction on fabricating a GLV
economically. This project would expand on Matiss's work by exploring the
construction of both affordable LDV and GLV measurement units, in particular
with an application to soundboards. Measuring musical instrument soundboard does
not require as wide a frequency bandwidth compared to other vibration analysis
applications. The Polytec PSV400  has bandwidth up to 1000kHz \cite{PSV400},
compare to soundboards where only up to 20kHz are consideration for measurement
as it is the limit of human hearing.

Primary application for scanning instruments is the ability to cross-reference
results with those obtained within simulation. 

Paper by Tahvanainen et al. \cite{HENNA} necessitated the measurements of eight
guitars, one with laser vribrometery, in order to obtain an agregate set of of
mosde shapes and top plate velocities. In this case research place at Yamaha
Research and Devlopment and  is a best case scenarion with respect to acces to
materials and tools. Low cost laser scanning virborrmetr would lower the barrier
in both cost and labour for projects of this kind making it more accessible.

Similarly, \cite{GUITAR_3D_SCAN} measured guitar mode shapes and frequencies for
comparison with those of a finite element model.

\cite{VIOLIN_3D_SCAN} Scanning laser vibrometer used for top plate measure of a
violin fabricated with bio-based composite materials.

The paper by Le Moyne \cite{MOYNE} provides an example of when non contact
measurements are essential in musical instrument restoration. In this case a
17th century Couchet harspichord was to be reconstructed.  In this case a
nearfield acoustical holography approach was taken. This was to avoid the need
for use of an impact hammer. Less intrusive methods of excitation are available,
such was used in \cite{HENNA} where a non-invasive excitation was achieved using
random noise in a speaker. The project would aim to provide a readily available
tool for musical instrument collection to use for the monitoring and restoration
of musical instruments.

All the above projects that use laser vibrometry, utilised scanning laser
vibrometers. A scanning LDV is typically in the price range of £100k as opposed
to £30k for a non-scanning LDV. The scanning LDV dratically reduces labour as
the need for repositioning the LDV is bypassed entirely. The project aims to
develop further methods that have been used in recent researched for obtaining
an affordable method to augment an LDV into a scanning LDV.

The paper of Hui et al. \cite{HUI} considers the addition of a linear scanning
unit paired with a laser vibrometer. In this paper high-resolution results were
obtained with a XY sliding table. A sliding table can be sourced for a
relatively low cost and is therefore within the scope of an affordable linear
scanning assembly.

Linear scanning using a standard two-dimensional plotter yielded high-resolution
results in \cite{HUI}. Such a measurement arrangement would allow for automated
scanning. The 3D printed impact hammer used for measurements  in \cite{RAU} can
also be combined with an actuator (e.g. a solenoid) for reproducible excitation.

\begin{figure}
\centering
\includegraphics[width=0.66\textwidth]{./img/linear-scanning-ldv.png}
\caption{LDV Mounted to plotter (labelled X-Y sliding table) \cite{HUI}}
\label{fig:ldv-linear-scanner}
\end{figure}

One large draw back of the approach in the paper by Hui et al is in scaling the
project. For the soundboard of an instrument such as a violin or a guitar the
scale of the sliding table assembly is relatively low cost and achievable. If a
piano or harpsichord sound board is to be considered the material costs and
labour cost in assembly are not insignificant. 

Another interesting approach to automated scanning would be through the use of
mirror galvanometers. A mirror galvanometer consists of a mirror attached to a
rotating spindle on its central axis. The spindle is attached to an assembly
similar to that of a stepper motor. The degree of rotation changes depending on
the voltage applied to the internal circuit. This arrangement allows for a laser
beam to be aimed at a point on a 2 dimensional plane using the rotational
movement \ref{fig:mirror-galv}. A common modern commercial application for a
pair of mirror galvanometers to be used with a laser is in animated laser light
displays. Given this commercial use, mirror galvanometers can be sourced for an
economic cost.

\begin{figure}
\centering
\includegraphics[width=0.66\textwidth]{./img/mirror-galv.pdf}
\caption{Setup of mirror galvanometer pair from \cite{GALVO} illustrating how
rotation of mirrors translates to project onto a 2D plane}
\label{fig:mirror-galv}
\end{figure}

The benefit of using mirror galvanometers a means of creating a scanning LDV is
that they would address the problem of scaling to larger soundboards. In order
to scan a larger area, the laser can simply be moved further away from the
soundboard. The mechanical travel is also less in comparison to the sliding
table, meaning scanning time would also be reduced. Such a system was used in
the thesis by Rau \cite{RAU_THESIS} to successfully measure the soundboard of a
guitar. One limitation of digitally controlling a mirror galvanometer is number
of potential points that can be measured. The bit-depth of the digital to analog
converter (DAC) that is used dictates the number of available grid points. In
the thesis by Rau \cite{RAU_THESIS} a 12-bit DAC was used, which provides a 4096
x 4096 grid of points that can be measured. A grid of that size would be
sufficient for a guitar or violin top plate, but in order to accommodate larger
soundboards, such as for piano or harpsichord, it would be beneficial to select
a DAC with a higher bit-depth.

Laser vibrometry and its application and practicalities within the field of
musical acoustics is still actively being explored, but it is not the only
method of non-contact measurement of  musical instruments. An array of MEMs
microphone has been considered as an option for non-contact economic modal
analysis. This is a promising approach for obtaining FRFs as outlined in
\cite{FARSHIDI2010755} and may be accurate enough for estimating material
properties. MEMs microphone arrays as a means of plate mode measurement is
demonstrated in \cite{VELSEN} stating that for a MEMs microphone array:

\begin{quote}
...low frequency domain eigenfrequencies of a plate-like structure can be
identified with high accuracy. Identification of the mode shapes and the damping
constants are shown not to be accurate.
\end{quote}

Since the focus of the project is on the derivation of mode shapes, as well as
mode frequencies, the usage of MEMs microphone arrays has not been considered.

\hypertarget{project-description}{%
\subsection{Project Description}\label{project-description}}

The project would aim to to look at two types of laser vibrometry methods.

\begin{itemize}
\tightlist
\item
  Laser Doppler Velocimetry
\item
  Grazing Laser Vibrometry
\end{itemize}

Each method has a base cost for the most simple instance, with extra costs
incurred from components with high tolerance and signal filtering.

\begin{figure}%
  \centering
  \subfloat[\centering LDV Block
  Diagram]{{\includegraphics[width=5cm]{./img/ldv-block.png} }}%
  \qquad
  \subfloat[\centering GLV Block
  Diagrams]{{\includegraphics[width=5cm]{./img/grv-block.png} }}%
  \caption{Block Diagrams of both LDV and GLV from \cite{MALAHS2015DESIGNOA}}%
  \label{fig:laser-block}%
\end{figure}

The initial step of the project would be to construct at the simplest version
for each vibrometer and test its accuracy. Depending on results, the fabrication
can then be refined or a second version can be created with higher quality
components. The benefits provided from these components can be quantified
against their cost and documented.

Goal of the project would be the design of an automated non-contact measurement
system for musical instrument soundboards. The system would aim to be economic
to fabricate in comparison to current commercially available laser vibrometers.

The project would focus on the creation of an open source, version controlled
framework for designing a low-cost, non-contact measurement system for musical
instrument soundboards consisting of:

\begin{itemize}
\tightlist
  \item
    Operation Manual Repository: An open source repository containing:
  \begin{itemize}
  \tightlist
    \item
      Instructions for self-assembly of measurement tools and experiment setup 
    \item
      CAD models for digital fabrication of components and PCB
    \item
      Control software for automation of a measurement rig
    \item
      Analysis software for recording, processing and analysing measurement
    data.
  \end{itemize}

\item
  Technical Documentation: A document containing: , justification of design
decisions, discussion of formulae required to obtain detain from the system, 
  \begin{itemize}
  \tightlist
    \item
      An outline of the design process
    \item
      Technical background and justifications for any design decision and
    components utilised
    \item
      Design modifications, their relative cost to the project and impact on
    results. 
  \end{itemize}
\end{itemize}


\hypertarget{materials}{%
\subsubsection{Materials}\label{materials}}

Listed below are materials the materials required categorised by their function
with a brief explanation of their purpose within the greater project. With
respect to costing, all prices given are given are an aggregate of relevant
suppliers recorded on \date{\today}. Suppliers used are listed before each
section. Guide prices here a listed to give a rough costing and do not include
additional costs such as customs and shipping. In some cases there could also be
a decrease in price when buying multiple units.

\hypertarget{general-electronics}{%
\paragraph{General electronics}\label{general-electronics}}

The proposed project will require a workshop space with facilities applicable to
all approaches of measurement techniques. In addition to access to general
electronic sundries (resistors, capacitors, potentiometers), a workshop would
require facilities to aid prototyping. Suppliers RS and Element 14 were used as
a source for guide prices:

\begin{itemize}
\tightlist
\item
  Bench Power Supply: Parametric power supply required for prototyping
  electronics when power requirements are still in flux. (Guide Price: £125)
\item
  Function generator: For applying a functional signal to a circuit for
  simulating input from sensors. (Guide Price: £160)
\item
  Oscilloscope: Signal measurement and testing. (Guide Price: £160)
\item
  Soldering Station: For soldering though-hole and surface mount components.
  (optionally) A reflow oven for soldering surface mount components to a pcb.
  Alternatively this step could be carried out from an external service. (Guide
  Price: £150)
\end{itemize}

\hypertarget{laser-components}{%
\paragraph{Laser Components}\label{laser-components}}

Components relevant to each vibrometer type are listed below. Suppliers RS and
Element 14 were used as a source for guide prices:

  \begin{itemize}
  \tightlist
  \item
    LM348 op-amps: used in the Transimpedance Amplifier as well as filtering
    circuits, and level shifting. As recommended in \cite{MALAHS2015DESIGNOA} an
    alternative would aso be sought to improve sensor bandwidth. (Guide Price:
    £0.70)
  \item
    650nm Laser: Bandwidth chosen in \cite{MALAHS2015DESIGNOA} to match the
    BPW34 photodiodes. The BPW3 is shown to have the best response in the
    Infrared range so a dual laser system, one for aiming one for measuring,
    would be desirable. (Guide Price: £20)
  \item
    BPW34 photodiodes: Used for detecting changes in the laser beam. Current is
    induced when the light strikes the surface. (Guide Price: £0.80)
  \item
    Analog to Digital Converter: The GLV from \cite{MALAHS2015DESIGNOA} was
    connected to a National Instruments myDAQ which provided enough resolution
    for processing data. As such, an ADC of similar specifications would be
    desirable. (Guide Price: £650)
  \item
    Linear Scanning Rig using either:
    \begin{enumerate}
    \tightlist
    \item
        An assembly of multiple stepper motors cascaded onto two sets of guide
        rails to for movement in 2 dimensions. These rigs can be purchased as a
        kit or assembled from separate parts if a suitable size is not
        available. This assembly can be seen in Figure
        \ref{fig:ldv-linear-scanner} labelled as ``X-Y Sliding Table'' (Guide
        Price: £400). They consist of: 
        \begin{itemize}
        \tightlist
        \item
          Stepper Motor: Motors and associated driver chips that can be
          controlled step-wise for accuracy.
        \item
          Micro-Controller Unit: Programmable computer chip that would allow for
          firmware to be written in order to control the steppers.
        \item
          Guide Rails: Lubricated aluminium rails to help guide each platform
          along a particular dimensions
        \item
          Aluminium Extrusion: Used as frame and platform for stepper motors and
          the vibrometer assembly.
        \end{itemize}
    \item
        A pair of mirror galvanometers, such as in \ref{fig:mirror-galv},
        controlled by a MCU. Consisting of: 
        \begin{itemize}
        \tightlist
        \item
            Mirror Galvanometer pair (Guide Price: £200)
        \item
            Micro-controller
        \item
            16-bit DAC
        \end{itemize}
    \end{enumerate}
  \end{itemize}

\hypertarget{optics}{%
\subparagraph{Optical Components}\label{optics}}

The LDV requires some additional optical components in order to function
correctly and constitute the bulk of the cost of fabrication. Prices for each
component are an aggregate of those listed for components on Edmund Optics,
AeroDiode and OptoShop.

\begin{itemize}
\tightlist
\item
  Beam Splitter: Allows for redirection of laser beam into two direction. (Guide
  Price: £160)
\item
  Mirror: for redirecting beams, Fibre optics are also a possible substitution.
  (Guide Price: £200)
\item
  Bragg Cell: (Optional) Also known as Acousto-optic Modulators. This component
  modulates the light frequency which produces a fringe pattern on the
  photodetector. The bragg cell is the most expensive component of the LDV and
  as suggested in \cite{MALAHS2015DESIGNOA}, the inclusion is not necessary if
  directional information is not required. The first design of the LDV would
  omit this component entirely. (Guide Price: £1500)
\item
  Focal Lens: (Optional) Should the laser beam need to be focused into a fibre
  optic cable, a beam splitter would be necessary. (Guide Price: £400)
\end{itemize}

\hypertarget{safety}{%
\subsubsection{Safety}\label{safety}}

Since the project will be working with lasers there are safety concerns. The
design in \cite{MALAHS2015DESIGNOA} was Category 3R laser with a maximum power
of 5mW at 3V.

\begin{quote}
Class 3R lasers are specified as to create no risk to skin and low risk to eyes
with power limit up to 5 mW [\cite{SAFETY}]
\end{quote}

Eye protection would need to be worn and appropriate signage displayed in any
workshop space.

The designs considered would initially use a commercial power supply unit or
battery power supply units. As such, power electronic design is not considered
as part of the design process and no high-voltage power electronic design will
be taking place.


\hypertarget{timeline}{%
\subsubsection{Timeline}\label{timeline}}

\subsubsection{Stage One: Measurement unit prototyping}

The goal of this stage is to create a LDV and GLV unit. The LDV would not
contain a Bragg Cell and the viability of the system can be tested against the
GLV.

\subsubsection{Stage Two: Linear Scanning}

Both LDV and GLV units will be tested when mounted on a linear scanning system.

Main goal is to create a means of stipulating geometry and asses each unit
against a simple rectangular plate.

\subsubsection{Stage Three: Documentation}

The final stage will hope to generate a document which will facilitate future
modal analysis Projects. Focus of this stage is the collation of a written
thesis, technical documentation and an open source hardware and software
repository, with a focus on software sustainability to make the project more
easily developed and reproducible in the future.

\printbibliography

\end{document}
