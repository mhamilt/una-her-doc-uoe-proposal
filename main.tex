% Options for packages loaded elsewhere
\PassOptionsToPackage{unicode}{hyperref} \PassOptionsToPackage{hyphens}{url}
%
\documentclass[ ]{article}
\usepackage{amsmath,amssymb}
\usepackage{iftex}
\ifPDFTeX
  \usepackage[T1]{fontenc}
  \usepackage[utf8]{inputenc}
  \usepackage{textcomp} % provide euro and other symbols
\else % if luatex or xetex
  \usepackage{unicode-math} % this also loads fontspec
  \defaultfontfeatures{Scale=MatchLowercase}
  \defaultfontfeatures[\rmfamily]{Ligatures=TeX,Scale=1}
\fi
\usepackage{lmodern}
\ifPDFTeX\else
  % xetex/luatex font selection
\fi

% Bibliography Package and Setup
\usepackage[sorting=none]{biblatex}
\addbibresource{bibliography.bib}

\usepackage{graphicx}
\usepackage{subfig}

% Use upquote if available, for straight quotes in verbatim environments
\IfFileExists{upquote.sty}{\usepackage{upquote}}{}
\IfFileExists{microtype.sty}{% use microtype if available
  \usepackage[]{microtype}
  \UseMicrotypeSet[protrusion]{basicmath} % disable protrusion for tt fonts
}{}
\makeatletter
\@ifundefined{KOMAClassName}{% if non-KOMA class
  \IfFileExists{parskip.sty}{%
    \usepackage{parskip}
  }{% else
    \setlength{\parindent}{0pt}
    \setlength{\parskip}{6pt plus 2pt minus 1pt}} }{% if KOMA class
  \KOMAoptions{parskip=half}}
\makeatother
\usepackage{xcolor}
\setlength{\emergencystretch}{3em} % prevent overfull lines
\providecommand{\tightlist}{%
  \setlength{\itemsep}{0pt}\setlength{\parskip}{0pt}}
\setcounter{secnumdepth}{-\maxdimen} % remove section numbering
\ifLuaTeX \usepackage{selnolig}  % disable illegal ligatures
\fi
\IfFileExists{bookmark.sty}{\usepackage{bookmark}}{\usepackage{hyperref}}
\IfFileExists{xurl.sty}{\usepackage{xurl}}{} % add URL line breaks if available
\urlstyle{same}
\hypersetup{ hidelinks, pdfcreator={LaTeX via pandoc}}

\title{Autmated Non-contact measurement system design for muscial instryment soundboards}
\author{Matthew Hamilton}
\date{\today}

\begin{document}

\begin{titlepage}
\maketitle

\begin{figure}
\centering
\includegraphics[width=0.66\textwidth]{./img/mode-shape.png}
\label{fig:mode-shape}
\end{figure}
\end{titlepage}

\hypertarget{overview}{%
\section{Overview}\label{overview}}

Modal analysis is a method to extrapolate frequency response, vibration patterns
of a surface \cite{SKRODZKA, MOYNE}, and can also provide a means of estimating
material properties \cite{DUCCESCHI2024109949}, which is important information
in the analysis of musical instruments \cite{FRITZ,MOYNE,ELEJABARRIETA}.

Risk to a musical instrument is prominent concern during analysis, in particular
to historic musical instruments. This has lead to the emergence of non-contact
methods for analysis, which are an appealing method for modal analysis for
fragile, historically important, and expensive instruments.

Laser vibrometry has long been a useful tool for acoustic analysis
\cite{SRIRIAM}, especially with respect to musical instruments. Laser vibrometry
provides a non-obtrusive means of obtaining accurate results, however it does
have a limitation in the form of cost. A modest laser doppler vibrometer (LDV)
cost in the order of €20k. In addition, estimation of material properties of
soundboards comes with an additional cost of Finite Element Method modelling
software. This is a cost that is difficult to justify for the usage it would see
by instrument collector and museums. The market for laser vibrometer is also
dominated by a handful of companies and it seems unlikely this cost will shrink
any time soon. Laser vibrometers are precision engineered for all vibration
analysis, a level accuracy unlikely to be utlilsed when applied to musical
instruments. 

The thesis by Malah \cite{MALAHS2015DESIGNOA} outlines such a system for a
Grazing Laser Vibrometer. The proposed project would expand Malah's work by
applying it to a Laser Doppler Vibrometer. In addition, the project will explore
automating the process of measurement and provide an open source framework for
analysing results.

\begin{figure}
  \centering
  \includegraphics[width=0.66\textwidth]{./img/ldv-setup.png}
  \caption{LDV Measurement Setup in \cite{MANSOUR}}
  \label{fig:ldv-setup}
\end{figure}


The barrier to entry for technical projects of this kind has been lowered over
the past few decades. What was once quite a technical undertaking is now more
achievable with readily available micro-controller units (MCUs) like the Arduino
or single-board computers (SBCs) like the Raspberry Pi, which can easily
interface with sensors and actuators to prototype haptic devices. 

\begin{figure}
\centering
\includegraphics[width=0.66\textwidth]{./img/guitar-mode-table.png}
\caption{Table of guitar body mode shapes derived from LDV in \cite{SKRODZKA}}
\label{fig:ldv-setup}
\end{figure}

\hypertarget{recent-research}{%
\subsection{Recent Research}\label{recent-research}}

This section will discuss some projects from previous years focused on the
topics of non-contact, automated modal analysis and making the process more
economically accessible. 

 The thesis by Matiss Malahs \cite{MALAHS2015DESIGNOA} researched low-cost laser
vibrometer system. The thesis provides an overview of interferometry, laser
doppler vibrometers and Grazing laser vibrometers as well as an instruction on
fabricating a GLV economically. The proposed project would aim to expand on
Matiss's work by exploring the construction of both affordable LDV and GLV
measurement units, in particular with an application to soundboards. Measuring
musical instrument soundboard does not require as wide a frequency bandwidth
compared to other vibration analysis applications. Frequency analysis for
soundboards generally considers up to 20kHz and a standard LDV would have a
bandwidth of up to 1000kHz \cite{PSV400}. The results obtain in the paper by
Malahs suggests the bandwidth of the LDV can be reduced to the benefit of lower
manufacturing costs.

The paper by Hui et al. \cite{HUI} considers the pairing of a linear scanning
unit with a laser vibrometer, which allowed for high-resolution results to be
obtained. This project would expand by applying the same methodology to an
economic laser vibrometer assembly as a measurement setup of this kind would
allow for automated scanning. The benefits of automating the measurement
process for modal analysis are two-fold. Firstly, automation would allow the
entire process to take place with minimum manual intervention, saving time in
labour and potentially time in carrying out measurements. Secondly, an automated
process would make measurements more easily reproducible. Multiple rounds of
measurements could be taken and the accuracy of the experimental setup could be
more easily tested. Such a measurement arrangement would allow for automated
scanning. The 3D printed impact hammer used for measurements  in \cite{RAU} can
also be combined with an actuator (e.g. a solenoid) for reproducible excitation.

\begin{figure}
\centering
\includegraphics[width=0.66\textwidth]{./img/linear-scanning-ldv.png}
\caption{LDV Mounted to plotter (labelled X-Y sliding table) \cite{HUI}}
\label{fig:ldv-linear-scanner}
\end{figure}


For instrument restoration such as demonstrated in \cite{MOYNE}, laser
vibrometry is a tool for being able to recreate an instrument. For such a
use-case, the project would aim to provide an estimation of the material
properties which can be used cross referenced with material properties found in
literature.

Laser vibrometry and its application and practicalities within the field of
musical acoustics is still actively being explored, but it is not the only
method of non-contact measurement of  musical instruments. An array of MEMs
microphone has been considered as an option for non-contact economic modal
analysis. This is a promising approach for obtaining FRFs as outlined in
\cite{FARSHIDI2010755} and may be accurate enough for estimating material
properties. MEMs microphone arrays as a means of plate mode measurement is
demonstrated in \cite{VELSEN} stating that for a MEMs microphone array:

\begin{quote}
...low frequency domain eigenfrequencies of a plate-like structure can be
identified with high accuracy. Identification of the mode shapes and the damping
constants are shown not to be accurate.
\end{quote}

Since the focus of the project is on derivation of mode frequencies and shapes
exploration specifically is not a goal of the project, the usage of MEMs
microphone arrays has not been considered.

\cite{RAU}

\hypertarget{project-description}{%
\subsection{Project Description}\label{project-description}}

The project would aim to to look at 2 types of laser vibrometry.

\begin{itemize}
\tightlist
\item
  Laser Doppler
\item
  Grazing Laser
\end{itemize}

Each method has a base cost for the most simple instance, with extra costs
incurred from components with high tolerance and signal filtering.

\begin{figure}%
  \centering
  \subfloat[\centering LDV Block
  Diagram]{{\includegraphics[width=5cm]{./img/ldv-block.png} }}%
  \qquad
  \subfloat[\centering GLV Block
  Diagrams]{{\includegraphics[width=5cm]{./img/grv-block.png} }}%
  \caption{Block Diagrams of both LDV and GLV from \cite{MALAHS2015DESIGNOA}}%
  \label{fig:laser-block}%
\end{figure}

The initial step of the project would be to construct at the simplest version
for each vibrometer and test its accuracy. Depending on results, the fabrication
can then be refined or a second version can be created with higher quality
components. The benefits provided from these components can be quantified
against their cost and documented.

Goal of the project would be the design of an automated non-contact measurement
system for musical instrument soundboards. The system would aim to be economic
to fabricate in comparison to current commercially available laser vibrometers.

The project would focus on the creation of an open source, version controlled
framework for designing a low-cost, non-contact measurement system for musical
instrument soundboards consisting of:

\begin{itemize}
\tightlist
  \item
    Operation Repository: An open source repository containing:

  \begin{itemize}
  \tightlist
    \item
      Instructions for self-assembly of measurement tools and experiment setup 
    \item
      CAD models for digital fabrication of components and PCB
    \item
      Control software for automation of a measurement rig
    \item
      Analysis software for recording, processing and analysing measurement
    data.
  \end{itemize}

\item
  Technical Documentation: A document containing: , justification of design
decisions, discussion of formulae required to obtain detain from the system, 
  \begin{itemize}
  \tightlist
    \item
      An outline of the design process
    \item
      Technical background and justifications for any design decision and
    components utilised
    \item
      Design modifications, their relative cost to the project and impact on
    results. 
  \end{itemize}
\end{itemize}


\hypertarget{materials}{%
\subsubsection{Materials}\label{materials}}

Listed below are materials the materials required categorised by their function
with a brief explanation of their purpose within the greater project. With
respect to costing, all prices given are given are an aggregate of relevant
suppliers recorded on \date{\today}. Suppliers used are listed before each
section. Guide prices here a listed to give a rough costing and do not include
additional costs such as customs and shipping. In some cases there could also be
a decrease in price when buying multiple units.

\hypertarget{general-electronics}{%
\paragraph{General electronics}\label{general-electronics}}

The proposed project will require a workshop space with facilities applicable to
all approaches of measurement techniques. In addition to access to general
electronic sundries (resistors, capacitors, potentiometers), a workshop would
require facilities to aid prototyping. Suppliers RS and Element 14 were used as
a source for guide prices:

\begin{itemize}
\tightlist
\item
  Bench Power Supply: Parametric power supply required for prototyping
  electronics when power requirements are still in flux. (Guide Price: £125)
\item
  Function generator: For applying a functional signal to a circuit for
  simulating input from sensors. (Guide Price: £160)
\item
  Oscilloscope: Signal measurement and testing. (Guide Price: £160)
\item
  Soldering Station: For soldering though-hole and surface mount components.
  (optionally) A reflow oven for soldering surface mount components to a pcb.
  Alternatively this step could be carried out from an external service. (Guide
  Price: £150)
\end{itemize}

\hypertarget{laser-components}{%
\paragraph{Laser Components}\label{laser-components}}

Components relevant to each vibrometer type are listed below. Suppliers RS and
Element 14 were used as a source for guide prices:

  \begin{itemize}
  \tightlist
  \item
    LM348 op-amps: used in the Transimpedance Amplifier as well as filtering
    circuits, and level shifting. As recommended in \cite{MALAHS2015DESIGNOA} an
    alternative would aso be sought to improve sensor bandwidth. (Guide Price:
    £0.70)
  \item
    650nm Laser: Bandwidth chosen in \cite{MALAHS2015DESIGNOA} to match the
    BPW34 photodiodes. The BPW3 is shown to have the best response in the
    Infrared range so a dual laser system, one for aiming one for measuring,
    would be desirable. (Guide Price: £20)
  \item
    BPW34 photodiodes: Used for detecting changes in the laser beam. Current is
    induced when the light strikes the surface. (Guide Price: £0.80)
  \item
    Analog to Digital Converter: The GLV from \cite{MALAHS2015DESIGNOA} was
    connected to a National Instruments myDAQ which provided enough resolution
    for processing data. As such, an ADC of similar specifications would be
    desirable. (Guide Price: £650)
  \item
    Linear Scanning Rig: An assembly of multiple stepper motors cascade onto two
    sets of rails, one for movement in 2 dimensions. These rigs can be purchased
    as a kit or assembled from separate parts if a suitable size is not
    available. This assembly can be seen in Figure \ref{fig:ldv-linear-scanner}
    labelled as ``X-Y Sliding Table'' (Guide Price: £400). They consist of: 
    \begin{itemize}
    \tightlist
    \item
      Stepper Motor: Motors and associated driver chips that can be controlled
      step-wise for accuracy.
    \item
      Micro-Controller Unit: Programmable computer chip that would allow for
      firmware to be written in order to control the steppers.
    \item
      Guide Rails: Lubricated aluminium rails to help guide each platform along
      a particular dimensions
    \item
      Aluminium Extrusion: Used as frame and platform for stepper motors and the
      vibrometer assembly.
    \end{itemize}
  \end{itemize}

\hypertarget{optics}{%
\subparagraph{Optical Components}\label{optics}}

The LDV requires some additional optical components in order to function
correctly and constitute the bulk of the cost of fabrication. Prices for each
component are an aggregate of those listed for components on Edmund Optics,
AeroDiode and OptoShop.

\begin{itemize}
\tightlist
\item
  Beam Splitter: Allows for redirection of laser beam into two direction. (Guide
  Price: £160)
\item
  Mirror: for redirecting beams, Fibre optics are also a possible substitution.
  (Guide Price: £200)
\item
  Bragg Cell: (Optional) Also known as Acousto-optic Modulators. This component
  modulates the light frequency which produces a fringe pattern on the
  photodetector. The bragg cell is the most expensive component of the LDV and
  as suggested in \cite{MALAHS2015DESIGNOA}, the inclusion is not necessary if
  directional information is not required. The first design of the LDV would
  omit this component entirely. (Guide Price: £1500)
\item
  Focal Lens: (Optional) Should the laser beam need to be focused into a fibre
  optic cable, a beam splitter would be necessary. (Guide Price: £400)
\end{itemize}

\hypertarget{safety}{%
\subsubsection{Safety}\label{safety}}

Since the project will be working with lasers there are safety concerns. The
design in \cite{MALAHS2015DESIGNOA} was Category 3R laser with a maximum power
of 5mW at 3V.

\begin{quote}
Class 3R lasers are specified as to create no risk to skin and low risk to eyes
with power limit up to 5 mW [\cite{SAFETY}]
\end{quote}

Eye protection would need to be worn and appropriate signage displayed in any
workshop space.

The designs considered would initially use a commercial power supply unit or
battery power supply units. As such, power electronic design is not considered
as part of the design process and no high-voltage power electronic design will
be taking place.


\hypertarget{timeline}{%
\subsubsection{Timeline}\label{timeline}}

\subsubsection{Stage One: Measurement unit prototyping}

The goal of this stage is to create a LDV and GLV unit. The LDV would not
contain a Bragg Cell and the viability of the system can be tested against the
GLV.

\subsubsection{Stage Two: Linear Scanning}

Both LDV and GLV units will be tested when mounted on a linear scanning system.

Main goal is to create a means of stipulating geometry and asses each unit
against a simple rectangular plate.

\subsubsection{Stage Three: Documentation}

The final stage will hope to generate a document which will facilitate future
modal analysis Projects. Focus of this stage is the collation of a written
thesis, technical documentation and an open source hardware and software
repository, with a focus on software sustainability to make the project more
easily developed and reproducible in the future.

\printbibliography

\end{document}
